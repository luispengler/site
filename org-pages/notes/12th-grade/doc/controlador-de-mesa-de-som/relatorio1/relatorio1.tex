\documentclass{article}
\title{Sistema de controle de mesa de som - Sistemas digitais e microcontrolados}
\date{}

\usepackage[utf8]{inputenc}
\usepackage[portuguese]{babel}
\usepackage[margin=3.5cm]{geometry}
\usepackage{amsmath}
\usepackage{physics}
\usepackage{titlesec}
\usepackage{graphicx}
\usepackage{wrapfig}
\usepackage{caption}
\usepackage{subcaption}
\usepackage[parfill]{parskip}
\usepackage[nottoc]{tocbibind}
\usepackage[backend=biber]{biblatex}
\addbibresource{/home/luispengler/drive/LinuxFabrik/Research/read/bib.bib}
\usepackage{authblk}
\author[1]{Giovanna Bughi}
\author[2]{Gustavo Ratier Cardoso}
\author[3]{João Vitor Medeiros}
\author[4]{Luís Spengler}
\affil[1,2,3,4]{Instituto Federal de Educação, Ciência e Tecnologia de Mato Grosso do Sul}

\graphicspath{{./docs/}}

\begin{document}
\maketitle

\tableofcontents

\medskip

\section{}
\subsection{Problema proposto}

\begin{displaymath}
\begin{array}{|c c c|c c c|}
INPUT & & & OUTPUT &\\
\hline
ChP & ChD & ChC & SP & SD & SC\\
\hline % Put a horizontal line between the table header and the rest.
0 & 0 & 0 & 0 & 0 & 0\\
0 & 0 & 1 & 0 & 0 & 1\\
0 & 1 & 0 & 0 & 1 & 0\\
0 & 1 & 1 & 0 & 1 & 0\\
1 & 0 & 0 & 1 & 0 & 0\\
1 & 0 & 1 & 1 & 0 & 0\\
1 & 1 & 0 & 1 & 0 & 0\\
1 & 1 & 1 & 1 & 0 & 0\\
\end{array}
\end{displaymath}

\section{Conclusão}

Por fim, pode-se concluir que todos os gates (portas) testados experimentalmente neste relatório estavam funcionando corretamente. Até mesmo sua implementação na criação de uma porta NAND funcionou conforme esperado.

\medskip

\end{document}
